% debut d'un fichier latex standard
\documentclass[a4paper,12pt,twoside]{article}
% Tout ce qui suit le symbole "%" est un commentaire
% Le symbole "\" désigne une commande LaTeX

% pour l'inclusion de figures en eps,pdf,jpg, png
\usepackage{graphicx}

% quelques symboles mathematiques en plus
\usepackage{amsmath}

% le tout en langue francaise
%\usepackage[francais]{babel}

% on peut ecrire directement les caracteres avec l'accent
%    a utiliser sur Linux/Windows (! dépend de votre éditeur !)
%\usepackage[utf8]{inputenc} % Pour TeXworks
%\usepackage[latin1]{inputenc} % Pour Kile
%\usepackage[T1]{fontenc}

%    a utiliser sur le Mac ???
%\usepackage[applemac]{inputenc}

%-----------------------------------------------------------------------------------------------------------------
% Choix du bon package en fonction de la compilation
% ----------------------------------------------------------------------------------------------------------------

\usepackage{babel} %indique qu'on veut travailler en francais, en r�glant les accents, les espaces ins�cables, etc. automatiquement
\usepackage{iftex}

\ifPDFTeX
	\usepackage[utf8]{inputenc} %input encoding
	\usepackage[T1]{fontenc} %font encoding with pdflatex
	\usepackage{lmodern}
\else
	\ifXeTeX
		\usepackage{fontspec} %input encoding
	\else 
		\usepackage{luatextra}
	\fi
	\defaultfontfeatures{Ligatures=TeX}
\fi

% pour l'inclusion de liens dans le document 
\usepackage[colorlinks,bookmarks=false,linkcolor=blue,urlcolor=blue]{hyperref}

\paperheight=297mm
\paperwidth=210mm

\setlength{\textheight}{235mm}
\setlength{\topmargin}{-1.2cm} % pour centrer la page verticalement
%\setlength{\footskip}{5mm}
\setlength{\textwidth}{16.5cm}
\setlength{\oddsidemargin}{0.0cm}
\setlength{\evensidemargin}{-0.3cm}

\pagestyle{plain}

% nouvelles commandes LaTeX, utilis\'ees comme abreviations utiles
\def \be {\begin{equation}}
\def \ee {\end{equation}}
\def \dd  {{\rm d}}

\newcommand{\mail}[1]{{\href{mailto:#1}{#1}}}
\newcommand{\ftplink}[1]{{\href{ftp://#1}{#1}}}
%
% latex SqueletteRapport.tex      % compile la source LaTeX
% xdvi SqueletteRapport.dvi &     % visualise le resultat
% dvips -t a4 -o SqueletteRapport.ps SqueletteRapport % produit un PostScript
% ps2pdf SqueletteRapport.ps      % convertit en pdf

% pdflatex SqueletteRapport.pdf    % compile et produit un pdf

% ======= Le document commence ici ======

\begin{document}

	% Le titre, l'auteur et la date
	\title{Particule charg\'ee dans un champ \'electromagn\'etique}
	\author{Coppex Aur\'elie H\'el\`ene, Ventura Vincent\\  % \\ pour fin de ligne
	{\small \mail{aurelie.coppex@epfl.ch, vincent.ventura@epfl.ch}}}
	\date{\today}\maketitle
	\tableofcontents % Table des matieres

	% Quelques options pour les espacements entre lignes, l'indentation 
	% des nouveaux paragraphes, et l'espacement entre paragraphes
	%\baselineskip=16pt
	%\parindent=0pt
	%\parskip=12pt



\section{Introduction} %------------------------------------------


\section{Calculs analytiques}

	Dans cette partie, tous les calculs sont faits avec les valeurs suivantes :
	la masse de la particule : $m = 1.6726 \cdot 10^{-27}$ kg, 
	la charge de la particule :	$q = 1.6022 \cdot 10^{-19}$ C ,
	sa position initiale : $\vec{v_0} = (v_{x0}, v_{y0})$.
	La particule est plongée dans un champ \'electrique uniforme $\vec{E} = E_0 \hat{z}$ et un champ magn\'etique uniforme $\vec{B} = B_0 \hat{y}$.


	\subsection{\'Equations du mouvement}
	
		On cherche tout d'abord \`a \'etablir les \'equations diff\'erencielles du mouvement sous la forme $\frac{d\bf y}{dt} = {\bf f(\bf{y}})$ avec ${\bf y} = (x, z, v_x, v_z)$. 
		Pour cela, on applique la $2^{eme}$ loi de Newton ($\sum{\bf F} = m \bf a$ ) et on la projette sur les axes x, y et z.
		
		Il en r\'esulte :
		
		\begin{equation}
			{\bf F_p  + F_L} = m \bf{a}
		\end{equation}
		
		La force de pesanteur est n\'egligeable pour une particule \'el\'ementaire. Il ne reste donc que la force de Lorentz qui est donn\'ee par : $\vec{F} = q(\vec{E}+\vec{v} \times \vec{B})$.
		
		Les projections sur les axes donnent:
		
		\begin{equation}
			{\bf(Ox):} -qB_0 \dot{z} = m \ddot{x}  \iff \ddot{x} = -\frac{qB_0}{m} \dot{z}
		\end{equation}
		
		\begin{equation}
			{\bf(Oy):}  0 = m \ddot{y} \iff \ddot{y} = 0
		\end{equation}
		
		\begin{equation}
			{\bf(Oz):}  q(E_0 + B_0\dot{x}) = m \ddot{z} \iff \ddot{z} = \frac{q}{m} (E_0+B_0 \dot{x}) 
		\end{equation}
		
		L'\'equation du mouvement s'\'ecrit donc :
		\begin{equation}
		\frac{d\bf y}{dt} = \begin{pmatrix}  \dot x \\ \dot z \\ -\frac{qB_0}{m} \dot{z} \\ \frac{q}{m} (E_0+B_0 \dot{x} \end{pmatrix}
		\end{equation}
		

	\subsection{\'Energie m\'ecanique et sa conservation}
	
	
	
	\subsection{Solution analytique}


\section{Simulations et Analyses}

	\subsection{\'Etudes de convergence de la position et de la vitesse sans friction}


	\subsection{\'Etude de convergence de l'erreur $|x(v=0) - x_{eq}|$}

		
	\subsection{\'Etudes de convergence avec friction et comparaison de la puissance et de la d\'eriv\'ee temporelle de l'\'energie m\'ecanique}

		
	\subsection{Cas du positron}
	
	


\section{Conclusions}



%-----------------------------------------------------------


\begin{thebibliography}{99}
 \bibitem{NdC}
 L. Villard avec la contribution de A. L\"auchli {\it Notes de cours Physique numérique I-II, version 20.1} (2020)
 
 \bibitem{Notes}
 L. Villard, Dr C. Sommariva {\it \'Enonc\'e de l'exercice 2} (2020)
 \url{https://moodle.epfl.ch/pluginfile.php/2839539/mod_resource/content/1/Exercice2_2020.pdf}
 
\end{thebibliography}

\end{document} %%%% THE END %%%%
