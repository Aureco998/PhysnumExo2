% debut d'un fichier latex standard
\documentclass[a4paper,12pt,twoside]{article}
% Tout ce qui suit le symbole "%" est un commentaire
% Le symbole "\" désigne une commande LaTeX

% pour l'inclusion de figures en eps,pdf,jpg, png
\usepackage{graphicx}

% quelques symboles mathematiques en plus
\usepackage{amsmath}

% le tout en langue francaise
%\usepackage[francais]{babel}

% on peut ecrire directement les caracteres avec l'accent
%    a utiliser sur Linux/Windows (! dépend de votre éditeur !)
%\usepackage[utf8]{inputenc} % Pour TeXworks
%\usepackage[latin1]{inputenc} % Pour Kile
%\usepackage[T1]{fontenc}

%    a utiliser sur le Mac ???
%\usepackage[applemac]{inputenc}

%-----------------------------------------------------------------------------------------------------------------
% Choix du bon package en fonction de la compilation
% ----------------------------------------------------------------------------------------------------------------

\usepackage{babel} %indique qu'on veut travailler en francais, en r�glant les accents, les espaces ins�cables, etc. automatiquement
\usepackage{iftex}

\ifPDFTeX
	\usepackage[utf8]{inputenc} %input encoding
	\usepackage[T1]{fontenc} %font encoding with pdflatex
	\usepackage{lmodern}
\else
	\ifXeTeX
		\usepackage{fontspec} %input encoding
	\else 
		\usepackage{luatextra}
	\fi
	\defaultfontfeatures{Ligatures=TeX}
\fi

% pour l'inclusion de liens dans le document 
\usepackage[colorlinks,bookmarks=false,linkcolor=blue,urlcolor=blue]{hyperref}

\paperheight=297mm
\paperwidth=210mm

\setlength{\textheight}{235mm}
\setlength{\topmargin}{-1.2cm} % pour centrer la page verticalement
%\setlength{\footskip}{5mm}
\setlength{\textwidth}{16.5cm}
\setlength{\oddsidemargin}{0.0cm}
\setlength{\evensidemargin}{-0.3cm}

\pagestyle{plain}

% nouvelles commandes LaTeX, utilis\'ees comme abreviations utiles
\def \be {\begin{equation}}
\def \ee {\end{equation}}
\def \dd  {{\rm d}}

\newcommand{\mail}[1]{{\href{mailto:#1}{#1}}}
\newcommand{\ftplink}[1]{{\href{ftp://#1}{#1}}}
%
% latex SqueletteRapport.tex      % compile la source LaTeX
% xdvi SqueletteRapport.dvi &     % visualise le resultat
% dvips -t a4 -o SqueletteRapport.ps SqueletteRapport % produit un PostScript
% ps2pdf SqueletteRapport.ps      % convertit en pdf

% pdflatex SqueletteRapport.pdf    % compile et produit un pdf

% ======= Le document commence ici ======

\begin{document}

	% Le titre, l'auteur et la date
	\title{Particule charg\'ee dans un champ \'electromagn\'etique}
	\author{Coppex Aur\'elie H\'el\`ene, Ventura Vincent\\  % \\ pour fin de ligne
	{\small \mail{aurelie.coppex@epfl.ch, vincent.ventura@epfl.ch}}}
	\date{\today}\maketitle
	\tableofcontents % Table des matieres

	% Quelques options pour les espacements entre lignes, l'indentation 
	% des nouveaux paragraphes, et l'espacement entre paragraphes
	%\baselineskip=16pt
	%\parindent=0pt
	%\parskip=12pt



\section{Introduction} %------------------------------------------


\section{Calculs analytiques}



	\subsection{Cas avec friction}
		\subsubsection{\'Equations du mouvement}

		

		\subsubsection{Puissance de la force de friction}


	\subsection{Cas sans friction}
	
		\subsubsection{Position d'\'equilibre}

		.

		\subsubsection{\'Energie potentielle}


		\subsubsection{Vitesse initiale}



\section{Simulations et Analyses}

	\subsection{\'Etudes de convergence de la position et de la vitesse sans friction}


	\subsection{\'Etude de convergence de l'erreur $|x(v=0) - x_{eq}|$}

		
	\subsection{\'Etudes de convergence avec friction et comparaison de la puissance et de la d\'eriv\'ee temporelle de l'\'energie m\'ecanique}

		
	\subsection{Cas du positron}
	
	


\section{Conclusions}



%-----------------------------------------------------------


\begin{thebibliography}{99}
 \bibitem{NdC}
 L. Villard avec la contribution de A. L\"auchli {\it Notes de cours Physique numérique I-II, version 20.1} (2020)
\end{thebibliography}

\end{document} %%%% THE END %%%%
