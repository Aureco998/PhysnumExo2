% debut d'un fichier latex standard
\documentclass[a4paper,12pt,twoside]{article}
% Tout ce qui suit le symbole "%" est un commentaire
% Le symbole "\" désigne une commande LaTeX

% pour l'inclusion de figures en eps,pdf,jpg, png
\usepackage{graphicx}

% quelques symboles mathematiques en plus
\usepackage{amsmath}

% le tout en langue francaise
%\usepackage[francais]{babel}

% on peut ecrire directement les caracteres avec l'accent
%    a utiliser sur Linux/Windows (! dépend de votre éditeur !)
%\usepackage[utf8]{inputenc} % Pour TeXworks
%\usepackage[latin1]{inputenc} % Pour Kile
%\usepackage[T1]{fontenc}

%    a utiliser sur le Mac ???
%\usepackage[applemac]{inputenc}

%-----------------------------------------------------------------------------------------------------------------
% Choix du bon package en fonction de la compilation
% ----------------------------------------------------------------------------------------------------------------

\usepackage{babel} %indique qu'on veut travailler en francais, en r�glant les accents, les espaces ins�cables, etc. automatiquement
\usepackage{iftex}

\ifPDFTeX
	\usepackage[utf8]{inputenc} %input encoding
	\usepackage[T1]{fontenc} %font encoding with pdflatex
	\usepackage{lmodern}
\else
	\ifXeTeX
		\usepackage{fontspec} %input encoding
	\else 
		\usepackage{luatextra}
	\fi
	\defaultfontfeatures{Ligatures=TeX}
\fi

% pour l'inclusion de liens dans le document 
\usepackage[colorlinks,bookmarks=false,linkcolor=black,urlcolor=blue, citecolor=black]{hyperref}

\paperheight=297mm
\paperwidth=210mm

\setlength{\textheight}{235mm}
\setlength{\topmargin}{-1.2cm} % pour centrer la page verticalement
%\setlength{\footskip}{5mm}
\setlength{\textwidth}{16.5cm}
\setlength{\oddsidemargin}{0.0cm}
\setlength{\evensidemargin}{-0.3cm}

\pagestyle{plain}

% nouvelles commandes LaTeX, utilis\'ees comme abreviations utiles
\def \be {\begin{equation}}
\def \ee {\end{equation}}
\def \dd  {{\rm d}}

\newcommand{\mail}[1]{{\href{mailto:#1}{#1}}}
\newcommand{\ftplink}[1]{{\href{ftp://#1}{#1}}}
%
% latex SqueletteRapport.tex      % compile la source LaTeX
% xdvi SqueletteRapport.dvi &     % visualise le resultat
% dvips -t a4 -o SqueletteRapport.ps SqueletteRapport % produit un PostScript
% ps2pdf SqueletteRapport.ps      % convertit en pdf

% pdflatex SqueletteRapport.pdf    % compile et produit un pdf

% ======= Le document commence ici ======

\begin{document}

	% Le titre, l'auteur et la date
	\title{Particule charg\'ee dans un champ \'electromagn\'etique}
	\author{Coppex Aur\'elie H\'el\`ene, Ventura Vincent\\  % \\ pour fin de ligne
	{\small \mail{aurelie.coppex@epfl.ch, vincent.ventura@epfl.ch}}}
	\date{\today}\maketitle
	\tableofcontents % Table des matieres

	% Quelques options pour les espacements entre lignes, l'indentation 
	% des nouveaux paragraphes, et l'espacement entre paragraphes
	%\baselineskip=16pt
	%\parindent=0pt
	%\parskip=12pt



\section{Introduction} %------------------------------------------

	Dans cet exercice, il est question d'une particule charg\'ee dans un champ \'electromagn\'etique subissant une force de Lorentz. 
	
	Le but est d'en \'etudier la trajectoire afin de pouvoir analyser les propri\'et\'es de stabilit\'e et de convergence des cinq sch\'emas suivants : 
	Euler explicite, Euler implicite, Euler-Cromer, Runge-Kutta d'ordre 2 et Boris Buneman \cite{NdC}.
	\newpage

\section{Calculs analytiques}

	Dans cette partie, tous les calculs sont faits avec les valeurs suivantes :
	
	la masse de la particule : $m = 1.6726 \cdot 10^{-27}$ kg, 
	
	la charge de la particule :	$q = 1.6022 \cdot 10^{-19}$ C ,
	
	sa position initiale : $\vec{v_0} = (v_{x0}, v_{y0})$.
	
	La particule est plongée dans un champ \'electrique uniforme $\vec{E} = E_0 \hat{z}$ et un champ magn\'etique uniforme $\vec{B} = B_0 \hat{y}$.


	\subsection{\'Equations du mouvement}
	
		On cherche tout d'abord \`a \'etablir les \'equations diff\'erencielles du mouvement sous la forme $\frac{d\bf y}{dt} = {\bf f(\bf{y}})$ avec ${\bf y} = (x, z, v_x, v_z)$. 
		Pour cela, on applique la $2^{eme}$ loi de Newton ($\sum{\bf F} = m \bf a$ ) et on la projette sur les axes x, y et z.
		
		Il en r\'esulte :
		
		\begin{equation}
			{\bf F_p  + F_L} = m \bf{a}
		\end{equation}
		
		La force de pesanteur est n\'egligeable pour une particule \'el\'ementaire. Il ne reste donc que la force de Lorentz qui est donn\'ee par : $\vec{F} = q(\vec{E}+\vec{v} \times \vec{B})$.
		
		Les projections sur les axes donnent:
		
		\begin{equation}\label{xpp}
			{\bf(Ox):} -qB_0 \dot{z} = m \ddot{x}  \iff \ddot{x} = -\frac{qB_0}{m} \dot{z}
		\end{equation}
		
		\begin{equation}
			{\bf(Oy):}  0 = m \ddot{y} \iff \ddot{y} = 0
		\end{equation}
		
		\begin{equation}\label{zpp}
			{\bf(Oz):}  q(E_0 + B_0\dot{x}) = m \ddot{z} \iff \ddot{z} = \frac{q}{m} (E_0+B_0 \dot{x}) 
		\end{equation}
		
		L'\'equation du mouvement s'\'ecrit donc :
		\begin{equation}
		\frac{d\bf y}{dt} = \begin{pmatrix}  \dot x \\ \dot z \\ -\frac{qB_0}{m} \dot{z} \\ \frac{q}{m} (E_0+B_0 \dot{x}) \end{pmatrix}
		\end{equation}
		

	\subsection{\'Energie m\'ecanique et sa conservation}
	
		L'\'energie m\'ecanique est d\'efinie comme : $E_{mec} = E_{cin} + E_{pot}$. Or, l'\'energie cin\'etique est connue et vaut $E_{cin} = \frac{1}{2} m {\vec{v}}^2$.
		Il ne reste donc plus qu'\`a calculer l'\'energie potentielle. Pour cela, la formule $E_{pot} = \int_P^O \vec{F}\, \mathrm d\vec{l}$ \cite{FT}. Ce qui donne :
	
			\begin{equation}
				E_{pot} = qE_0 (z_0-z)
			\end{equation}
		
		Ainsi, l'\'equation suivante est obtenue :
	
			\begin{equation}
				E_{mec} = \frac{1}{2}m(v_x^2+v_z^2) + qE_0(z-z_o)
			\end{equation}
		
		Pour montrer que cette \'energie est conserv\'ee, il faut en calculer la d\'eriv\'ee:
	
			\begin{equation}
				\frac{dE_{mec}}{dt} = \frac{1}{2} m(2\dot{x}\ddot{x} + 2\dot{z}\ddot{z}) - 	qE_0\dot{z} 
			\end{equation}
		En appliquant les équations (\ref{xpp}) et (\ref{zpp}), on trouve:
	
			\begin{equation}
				\frac{dE_{mec}}{dt}= -qB_0\dot{z}\dot{x}+qE_0\dot{z}+qB_0\dot{x}\dot{z}-qE_0\dot{z} = 0
			\end{equation}
	
		Ce qui prouve que l'\'energie m\'ecanique est conserv\'ee.
	
	\subsection{Solution analytique}
	
		Les conditions initiales sont les suivantes : $\vec{x}(0) = 0$ et $ \vec{v}(0) = v_0\hat{z}$.
		
		En int\'egrant les \'equations (\ref{xpp}) et (\ref{zpp}), il est possible de trouver $\dot{x}$ et $\dot{z}$ :
		
			\begin{equation}\label{aa}
				\left \{
				\begin{array}{r c l}
					\dot{x} & = & -\frac{qB_0}{m}z + C_1  \\
					\dot{z} & = & \frac{q}{m} (E_0t + B_0x) + C_2\\
				\end{array}
				\right .
			\end{equation}
		
		En appliquant la condition initiale $ \vec{v}(0) = v_0\hat{z}$ à (\ref{aa}), on obtient $C_1 = \frac{qB_0}{m}z_0 $ et $C_2 = v_0 -\frac{q}{m}B_0x_0$.

			
		En les appliqant dans les \'equations de la vitesse, il en ressort :
				
			\begin{equation}\label{v}
				\left \{
				\begin{array}{r c l}
					v_x(t) & = & \frac{qB_0}{m}(z_0 - z)\\
					v_z(t) & = & v_0 + \frac{qB_0}{m}(x-x_0) + \frac{qE_0}{m}t\\
				\end{array}
				\right .
			\end{equation}
			
		\`A partir desquelles (\ref{v}) il est possible de trouver les \'equations de l'acc\'eleration:
		
			\begin{equation} \label{diff}
				\left \{
				\begin{array}{r c l}
					\ddot{x} - \frac{q^2B_0^2}{m^2}x & = & \frac{q^2B_0E_0}{m^2}t-\frac{qB_0 }{m}v_0 - \frac{q^2B_0^2}{m^2}x_0\\
					
					\ddot{z} -\frac{q^2B_0E_0}{m^2}z& = &  \frac{qE_0}{m} -\frac{q^2B_0E_0}{m^2}z_0\\
				\end{array}
				\right .
			\end{equation}
	
	En r\'esolvant les \'equations diff\'erentielles du 2\`eme ordre (\ref{diff}), on trouve la solution analytique suivante :
	
		\begin{equation} 
			\left \{
			\begin{array}{r c l}
				x(t) & = & \frac{qB_0v_0}{m} \cos(\frac{m}{qB_0}t) + \frac{E_0m}{qB_0^2} \sin(\frac{qB_0}{m}t) -\frac{E_0}{B_0}t - \frac{mv_0}{qB_0} \\
				v_x(t) & = & - \sin(\frac{qB_0}{m}t) + \frac{E_0}{B_0} \cos(\frac{qB_0}{m}t) -\frac{E_0}{B_0} \\
				z(t) & = & -\frac{mE_0}{qB_0^2} \cos(\frac{qB_0}{m}t) + \frac{mv_0}{qB_0} \sin(\frac{qB_0}{m}t) + \frac{mE_0}{qB_0^2}\\
				v_z(t) & = & \frac{E_0}{B_0} \sin(\frac{qB_0}{m}t) + v_0\cos(\frac{qB_0}{m}t)
			\end{array}
			\right .
		\end{equation}
	



\section{Simulations et Analyses}

	


\section{Conclusions}



%-----------------------------------------------------------


\begin{thebibliography}{99}
 \bibitem{NdC}
 L. Villard avec la contribution de A. L\"auchli {\it Notes de cours Physique numérique I-II, version 20.1} (2020)
 
 \bibitem{Notes}
 L. Villard, Dr C. Sommariva {\it \'Enonc\'e de l'exercice 2} (2020)
 \url{https://moodle.epfl.ch/pluginfile.php/2839539/mod_resource/content/1/Exercice2_2020.pdf}
 
 \bibitem{FT}
 VEREIN SCHWEIZERISCHER MATHEMATIK- UND PHYSIKLEHRER  et al. {\it Formulaires et tables : mathématiques, physique, chimie} Editions G d'Encre, 2015. ISBN 978-2-940501-41-0
 
\end{thebibliography}

\end{document} %%%% THE END %%%%
