% debut d'un fichier latex standard
\documentclass[a4paper,12pt,twoside]{article}
% Tout ce qui suit le symbole "%" est un commentaire
% Le symbole "\" désigne une commande LaTeX

% pour l'inclusion de figures en eps,pdf,jpg, png
\usepackage{graphicx}

% quelques symboles mathematiques en plus
\usepackage{amsmath}

% le tout en langue francaise
%\usepackage[francais]{babel}

% on peut ecrire directement les caracteres avec l'accent
%    a utiliser sur Linux/Windows (! dépend de votre éditeur !)
%\usepackage[utf8]{inputenc} % Pour TeXworks
%\usepackage[latin1]{inputenc} % Pour Kile
%\usepackage[T1]{fontenc}

%    a utiliser sur le Mac ???
%\usepackage[applemac]{inputenc}

%-----------------------------------------------------------------------------------------------------------------
% Choix du bon package en fonction de la compilation
% ----------------------------------------------------------------------------------------------------------------

\usepackage[francais]{babel} %indique qu'on veut travailler en francais, en r�glant les accents, les espaces ins�cables, etc. automatiquement
\usepackage{iftex}

\ifPDFTeX
	\usepackage[utf8]{inputenc} %input encoding
	\usepackage[T1]{fontenc} %font encoding with pdflatex
	\usepackage{lmodern}
\else
	\ifXeTeX
		\usepackage{fontspec} %input encoding
	\else 
		\usepackage{luatextra}
	\fi
	\defaultfontfeatures{Ligatures=TeX}
\fi

% pour l'inclusion de liens dans le document 
\usepackage[colorlinks,bookmarks=false,linkcolor=black,urlcolor=blue, citecolor=black]{hyperref}

\paperheight=297mm
\paperwidth=210mm

\setlength{\textheight}{235mm}
\setlength{\topmargin}{-1.2cm} % pour centrer la page verticalement
%\setlength{\footskip}{5mm}
\setlength{\textwidth}{16.5cm}
\setlength{\oddsidemargin}{0.0cm}
\setlength{\evensidemargin}{-0.3cm}

\pagestyle{plain}

% nouvelles commandes LaTeX, utilis\'ees comme abreviations utiles
\def \be {\begin{equation}}
\def \ee {\end{equation}}
\def \dd  {{\rm d}}

\newcommand{\mail}[1]{{\href{mailto:#1}{#1}}}
\newcommand{\ftplink}[1]{{\href{ftp://#1}{#1}}}
%
% latex SqueletteRapport.tex      % compile la source LaTeX
% xdvi SqueletteRapport.dvi &     % visualise le resultat
% dvips -t a4 -o SqueletteRapport.ps SqueletteRapport % produit un PostScript
% ps2pdf SqueletteRapport.ps      % convertit en pdf

% pdflatex SqueletteRapport.pdf    % compile et produit un pdf

% ======= Le document commence ici ======

\begin{document}

	% Le titre, l'auteur et la date
	\title{Particule charg\'ee dans un champ \'electromagn\'etique}
	\author{Coppex Aur\'elie H\'el\`ene, Ventura Vincent\\  % \\ pour fin de ligne
	{\small \mail{aurelie.coppex@epfl.ch, vincent.ventura@epfl.ch}}}
	\date{\today}\maketitle
	\tableofcontents % Table des matieres
\newpage
	% Quelques options pour les espacements entre lignes, l'indentation 
	% des nouveaux paragraphes, et l'espacement entre paragraphes
	%\baselineskip=16pt
	%\parindent=0pt
	%\parskip=12pt



\section{Introduction} %------------------------------------------

	Dans le domaine de la physique, il est possible de d\'ecrire de nombreux phénomènes du monde. Pour cela, il est souvent n\'ecessaire de recourir au calcul num\'erique pour r\'esoudre des \'equations qui ne peuvent pas \^etre calcul\'ees analytiquement. Il existe plusieurs m\'ethodes pour les r\'esoudre, dont les int\'egrateurs. Il faut cependant savoir les choisir en fonction du probl\`emes trait\'es et pour cela, il faut les conna\^itre.

	Dans cet exercice, il est question d'une particule charg\'ee dans un champ \'electromagn\'etique subissant une force de Lorentz. 
	
	Le but est d'en \'etudier la trajectoire afin de pouvoir analyser les propri\'et\'es de stabilit\'e et de convergence des cinq sch\'emas suivants : 
	Euler explicite, Euler implicite, Euler-Cromer, Runge-Kutta d'ordre 2 et Boris Buneman \cite{NdC}.


\section{Calculs analytiques}

	\noindent Dans cette partie, tous les calculs sont faits avec les valeurs suivantes \cite{Notes} :
	
	\noindent la masse de la particule : $m = 1.6726 \cdot 10^{-27}$ kg, 
	
	\noindent charge de la particule :	$q = 1.6022 \cdot 10^{-19}$ C ,
	
	\noindent sa position initiale : $\vec{v_0} = (v_{x0}, v_{y0})$.
	
	\noindent La particule est plongée dans un champ \'electrique uniforme $\vec{E} = E_0 \hat{z}$ et un champ magn\'etique uniforme $\vec{B} = B_0 \hat{y}$. 


	\subsection{\'Equations du mouvement}
	
		\noindent On cherche tout d'abord \`a \'etablir les \'equations diff\'erencielles du mouvement sous la forme $\frac{d\bf y}{dt} = {\bf f(\bf{y}})$ avec ${\bf y} = (x, z, v_x, v_z)$. 
		Pour cela, on applique la $2^{eme}$ loi de Newton ($\sum{\bf F} = m \bf a$ ) et on la projette sur les axes x, y et z.
		
		\noindent Il en r\'esulte :
		
		\begin{equation}
			{\bf F_p  + F_L} = m \bf{a}
		\end{equation}
		
		\noindent Avec ${\bf F_p}$, la force de pesanteur et ${\bf F_L}$, la force de Lorentz.
		
		\noindent La force de pesanteur est n\'egligeable pour une particule \'el\'ementaire. Il ne reste donc que la force de Lorentz qui est donn\'ee par : ${\bf F_L} = q({\bf E}+{\bf v} \times {\bf B})$.
		
		\noindent Les projections sur les axes donnent:
		
		\begin{equation}\label{xpp}
			{\bf(Ox)}: -qB_0 \dot{z} = m \ddot{x}  \iff \ddot{x} = -\frac{qB_0}{m} \dot{z}
		\end{equation}
		
		\begin{equation}
			{\bf(Oy)}:  0 = m \ddot{y} \iff \ddot{y} = 0
		\end{equation}
		
		\begin{equation}\label{zpp}
			{\bf(Oz)}:  q(E_0 + B_0\dot{x}) = m \ddot{z} \iff \ddot{z} = \frac{q}{m} (E_0+B_0 \dot{x}) 
		\end{equation}
		
	\noindent	L'\'equation du mouvement s'\'ecrit donc :
		\begin{equation}
		\frac{d\bf y}{dt} = \begin{pmatrix}  \dot x \\ \dot z \\ -\frac{qB_0}{m} \dot{z} \\ \frac{q}{m} (E_0+B_0 \dot{x}) \end{pmatrix}
		\end{equation}
		

	\subsection{\'Energie m\'ecanique et sa conservation}
	
		\noindent L'\'energie m\'ecanique est d\'efinie comme : $E_{mec} = E_{cin} + E_{pot}$. Or, l'\'energie cin\'etique est connue et vaut $E_{cin} = \frac{1}{2} m {\vec{v}}^2$.
		Il ne reste donc plus qu'\`a calculer l'\'energie potentielle. Pour cela, on utilise la formule $E_{pot} = \int_P^O \vec{F}\, \mathrm d\vec{l}$ \cite{FT}. Ce qui donne :
	
			\begin{equation}\label{E_p}
				E_{pot} = qE_0 (z_0-z)
			\end{equation}
		
		\noindent Ainsi, avec (\ref{E_p}) l'\'equation suivante est obtenue :
	
			\begin{equation}
				E_{mec} = \frac{1}{2}m(v_x^2+v_z^2) + qE_0(z_0 - z)
			\end{equation}
		
		\noindent Pour montrer que cette \'energie est conserv\'ee, il faut en calculer la d\'eriv\'ee:
	
			\begin{equation} \label{dE}
				\frac{dE_{mec}}{dt} = \frac{1}{2} m(2\dot{x}\ddot{x} + 2\dot{z}\ddot{z}) - 	qE_0\dot{z} 
			\end{equation}
		En appliquant les équations (\ref{xpp}) et (\ref{zpp}) \`a (\ref{dE}), on trouve:
	
			\begin{equation}
				\frac{dE_{mec}}{dt}= -qB_0\dot{z}\dot{x}+qE_0\dot{z}+qB_0\dot{x}\dot{z}-qE_0\dot{z} = 0
			\end{equation}
	
		\noindent Ce qui prouve que l'\'energie m\'ecanique est conserv\'ee.
	
	\subsection{Solution analytique}
	
		\noindent Les conditions initiales sont les suivantes : $\vec{x}(0) = 0$ et $ \vec{v}(0) = v_0\hat{z}$ \cite{Notes}.
		
		\noindent En int\'egrant les \'equations (\ref{xpp}) et (\ref{zpp}), il est possible de trouver $\dot{x}$ et $\dot{z}$ :
		
			\begin{equation}\label{aa}
				\left \{
				\begin{array}{r c l}
					\dot{x} & = & -\frac{qB_0}{m}z + C_1  \\
					\dot{z} & = & \frac{q}{m} (E_0t + B_0x) + C_2\\
				\end{array}
				\right .
			\end{equation}
		
		\noindent En utilisant la condition initiale $ \vec{v}(0) = v_0\hat{z}$ à (\ref{aa}), on obtient $C_1 = \frac{qB_0}{m}z_0 $ et $C_2 = v_0 -\frac{q}{m}B_0x_0$.

			
		\noindent En les appliquant dans les \'equations de la vitesse (\ref{aa}), il en ressort :
				
			\begin{equation}\label{v}
				\left \{
				\begin{array}{r c l}
					v_x(t) & = & \frac{qB_0}{m}(z_0 - z)\\
					v_z(t) & = & v_0 + \frac{qB_0}{m}(x-x_0) + \frac{qE_0}{m}t\\
				\end{array}
				\right .
			\end{equation}
			
		\noindent \`A partir de (\ref{v}), que l'on remplace dans (\ref{xpp}) et (\ref{zpp}), il est possible de trouver les \'equations de l'acc\'eleration:
		
			\begin{equation} \label{diff}
				\left \{
				\begin{array}{r c l}
					\ddot{x} - \frac{q^2B_0^2}{m^2}x & = & \frac{q^2B_0E_0}{m^2}t-\frac{qB_0 }{m}v_0 - \frac{q^2B_0^2}{m^2}x_0\\
					
					\ddot{z} -\frac{q^2B_0E_0}{m^2}z& = &  \frac{qE_0}{m} -\frac{q^2B_0E_0}{m^2}z_0\\
				\end{array}
				\right .
			\end{equation}
	
	\noindent En r\'esolvant les \'equations diff\'erentielles du 2\`eme ordre (\ref{diff}), on trouve la solution analytique suivante :
	
		\begin{equation} 
			\left \{
			\begin{array}{r c l}
				x(t) & = & \frac{qB_0v_0}{m} \cos \left(\frac{m}{qB_0}t\right) + \frac{E_0m}{qB_0^2} \sin \left(\frac{qB_0}{m}t\right) -\frac{E_0}{B_0}t - \frac{mv_0}{qB_0} \\
				v_x(t) & = & - v_0 \sin\left(\frac{qB_0}{m}t \right) + \frac{E_0}{B_0} \cos \left(\frac{qB_0}{m}t\right) -\frac{E_0}{B_0} \\
				z(t) & = & -\frac{mE_0}{qB_0^2} \cos\left(\frac{qB_0}{m}t\right) + \frac{mv_0}{qB_0} \sin\left(\frac{qB_0}{m}t\right) + \frac{mE_0}{qB_0^2}\\
				v_z(t) & = & \frac{E_0}{B_0} \sin\left(\frac{qB_0}{m}t\right) + v_0\cos\left(\frac{qB_0}{m}t\right)
			\end{array}
			\right .
		\end{equation}
	



\section{Simulations et Analyses}

	\subsection{Cas sans champ \'electrique mais avec un champ magn\'etque}
		\noindent Dans cette partie, les simulations se font avec les valeurs suivantes \cite{Notes} :
		
		\noindent L'intensit\'e du champ \'electrique $E_0=0$,
		
		\noindent l'intensit\'e du champ magn\'etique $B_0=5$T,
		
		\noindent les conditions initiales  $\vec{x}(0) = 0$ et $ \vec{v}(0) = v_0\vec{e_z}$,
		
		\noindent la vitesse initiale $v_0 = 4 \cdot 10^5$ m/s et 
	
		\noindent le temps final $t_{fin} = \frac{10 \pi}{\omega} = 6.5593 \cdot 10^{-8}$ s, le temps n\'ecessaire pour que le proton effectue 5 p\'eriodes de rotation. 
		
		
		\subsubsection{Convergence num\'erique de l'erreur sur la position et la vitesse}
		
		\begin{figure}[h]
			\begin{minipage}[c]{.46\linewidth}
				\centering
				\includegraphics[scale=0.6]{Convergence_erreur_position_tous_integrateurs.png}
				\caption{\em\label{Fig:Convergence erreur position} Convergence de l'erreur sur la position en $\Delta t$}
			\end{minipage}
			\hfill%
			\begin{minipage}[c]{.46\linewidth}
				\centering
				\includegraphics[scale = 0.6]{Convergence_erreur_vitesse_tous_integrateurs.png}
				\caption{\em\label{Fig: Convergence erreur vitesse} Convergence de l'erreur sur la vitesse en $\Delta t$}
			\end{minipage}
		\end{figure}

		\noindent Dans la figure \ref{Fig:Convergence erreur position}, une simulation de l'erreur maximale locale sur la position a \'et\'e faite. C'est donc l'erreur num\'erique qui est observ\'ee. On repr\'esente les convergences selon les 5 int\'egrateurs : Euler explicite (E), Euler implicite (EI), Euler-Cromer (EC), Runge-Kutta d'ordre 2 (RK2) et Boris-Buneman (BB). On s'int\'eresse maintenant \`a leur ordre de convergence. Ce dernier est d\'efini ici comme \'etant la pente de la courbe form\'ee par l'erreur lorsque $\Delta t$ est petit. Une fonction a \'et\'e cr\'e\'ee dans le fichier ParameterScan.m pour calculer ces pentes. Les résultats sont les suivants.
		
		\noindent Pour Euler implicite, les ordres de convegence de la position et de la vitesse sont les m\^emes et valent 1. Il en va de m\^eme pour Euler implicite.
		
		\noindent Euler-Cromer est \'egalement d'ordre 1 pour la position et la vitesse, ce qui se confirme par la figure \ref{Fig:Convergence erreur position} car on voit que les trois droites des premiers int\'egrateurs sont parall\`eles. 
		
		\noindent Runge-Kutta et Boris-Buneman sont d'ordre 2, ce qui est confirmé par les deux droites parall\`eles sur la figure \ref{Fig: Convergence erreur vitesse}.
		
		
		
		

		\subsubsection{Non-conservation de l'\'energie m\'ecanique}
		
		
	\begin{figure}[h]
		\begin{minipage}[c]{.46\linewidth}
			\centering
			\includegraphics[scale = 0.6]{D_E_mec_E.png}
			\caption{\em\label{Fig:Emec E} Variation de l'\'energie m\'ecanique selon Euler explicite en $\Delta t$}
		\end{minipage}
		\hfill%
		\begin{minipage}[c]{.46\linewidth}
			\centering
			\includegraphics[scale = 0.6]{D_E_mec_RK2.png}
			\caption{\em\label{Fig: Emec RK2} Variation de l'\'energie m\'ecanique selon Runge-Kutta 2 en $\Delta t$}
		\end{minipage}
	\end{figure}

		\noindent La figure \ref{Fig:Emec E} montre que l'int\'egrateur d'Euler explicite est instable car, pour des petits pas de temps, on voit clairement que $\Delta E_{mec}$ croît exponentiellement. 
		En ce qui concerne Runge-Kutta, on voit sur la figure \ref{Fig: Emec RK2} que la variation de l'\'energie m\'ecanique croît fortement pour des petits pas de temps, elle est donc \'egalement instable.
		
				\begin{figure}[h]
				\begin{minipage}[c]{.46\linewidth}
					\centering
					\includegraphics[scale = 0.6]{D_E_mec_EC_n=1_2_3.png}
					\caption{\em\label{Fig:Emec EC} Variation de l'\'energie m\'ecanique selon Euler-Cromer en $\Delta t$ pour des petits nombres de pas }
				\end{minipage}
				\hfill%
				\begin{minipage}[c]{.46\linewidth}
					\centering
					\includegraphics[scale = 0.6]{D_E_mec_Ec_n=10_20_50_100.png}
					\caption{\em\label{Fig: Emec ECC} Variation de l'\'energie m\'ecanique selon Euler-Cromer en $\Delta t$ pour des plus grands nombres de pas}
				\end{minipage}
			\end{figure}
		
		Dans la figure \ref{Fig:Emec EC}, qui concerne l'int\'egrateur d'Euler-Cromer, il est observable qu'avec un pas de temps de 3 l'\'energie est conserv\'ee, alors que si il est plus petit que 3, l'\'energie m\'ecanique n'est pas conserv\'ee. Cela est d\^u au fait qu'Euler-Cromer n'est stable que si $w_c \delta t <2$. Avec $w_c = \frac{qB_0}{m}$, la vitesse de rotation du proton. Ce qui donne le nombre de pas limite $n > 2.3948$, ce qui est bien visible dans la figure \ref{Fig:Emec EC}.
		
		Dans le graphique \ref{Fig: Emec ECC}, on voit que plus le pas de temps est grand, plus l'\'energie est bien conserv\'ee, la courbe oscille autour de 0 et montre donc que c'est un sch\'ema stable quand le pas de temps minimum est respect\'e.
		
		\begin{figure}[h]
			\begin{minipage}[c]{.46\linewidth}
				\centering
				\includegraphics[scale = 0.6]{D_E_mec_BB.png}
				\caption{\em\label{Fig:Emec BB} Variation de l'\'energie m\'ecanique selon Boris Buneman en $\Delta t$ }
			\end{minipage}
			\hfill%
			\begin{minipage}[c]{.46\linewidth}
				\centering
				\includegraphics[scale = 0.6]{D_E_mec_EI.png}
				\caption{\em\label{Fig: Emec EI2} Variation de l'\'energie m\'ecanique selon Euler implicite et Euler explicite  en $\Delta t$}
			\end{minipage}
		\end{figure}
	
	La figure \ref{Fig:Emec BB} montre que l'int\'egrateur de Boris-Buneman conserve exactement l'\'energie m\'ecanique, aux arrondis pr\`es, pour tous les pas de temps et, par cons\'equent, pour toute valeur de $\Delta t$.
	
	Et la figure \ref{Fig: Emec EI2} est faite  afin de pouvoir comparer les deux int\'egrateurs d'Euler explicite et implicite. On voit nettement que le sch\'ema implicite est bien plus stable que l'explicite, cependant il d\'cro\^it contrairement \`a celui explicite. Cela est d\^u au fait qu'il dissipe de l'\'energie, ce qui est ce \`a quoi on s'attend, comme dit dans le polycopi\'e (\cite{NdC}, p.34). Cette dissipation est d'origine num\'erique et non pas physique.

		
	
\subsection{Cas avec un champ \'electrique et un champ magn\'etique : d\'erive $\Vec{E}\times\Vec{B}$}
\noindent Dans cette partie il est consider\'e que la valeur $E_0 \ne 0$, mais que $E_0 = 10^5$Vm$^{-1}$ \cite{Notes}. Les autres valeurs restent comme dans l'exercice pr\'ec\'edent. Les sch\'emas utilis\'ees sont Runge-Kutta d'ordre 2 et Boris-Buneman.
	\subsubsection{\'Etude de convergence en $\Delta t$}
	\noindent En impl\'ementent les solutions analytiques dans le fichier ParameterScan.m de Matlab, la figure repr\'esentant le maximum de l'erreur de la position en fonction de $\Delta t$ est obtenue. Nous obtenons des droites pour Runge-Kutta d'ordre 2 et pour Boris-Buneman, que l'on peut observer sur la figure \ref{Fig: Conv Pos}, ce qu'implique une convergence lin\'eaire en fonction de $\Delta t$. En utilisant un fit lin\'eaire on peut d\'eterminer l'ordre de convergence des deux sch\'emas, qui est de 2 ($y = 2x + 32$ et $y = 2x + 32.5$), ce qui est facilement v\'erifiable sur la figure \ref{Fig: Fit Lin Pos}.
	
	
	\begin{figure}[h]
				\begin{minipage}[c]{.46\linewidth}
					\centering
					\includegraphics[scale = 0.6]{conv_pos_RK_BB.png}
					\caption{\em\label{Fig: Conv Pos} Etude de la convergence de l'erreure maximale sur la position selon $\Delta t$}
				\end{minipage}
				\hfill%
				\begin{minipage}[c]{.46\linewidth}
					\centering
					\includegraphics[scale = 0.6]{fit_lin_pos.png}
					\caption{\em\label{Fig: Fit Lin Pos} Fit lin\'eaire de l'erreure sur la position en $\Delta t$}
				\end{minipage}
			\end{figure}
			
			
	\subsubsection{Conservation de l'\'energie}
	\noindent En prenant en compte le champ \'electrique, nous avons une nouvelle force \`a consid\'erer. La charge $q$ de la particule \'etant positive, on a $\Vec{F_E} = qE_0\Vec{e_z}$, et donc la force est orient\'ee dans la m\^eme direction que le champ \'electrique, ce qui implique que l'\'energie potentielle est n\'egative selon l'axe $\Vec{e_z}$ d'apr\`es la formule $\Vec{E} = -\int \Vec{F}d\Vec{z}$ \cite{FT}. On peut donc exprimer l'\'energie m\'ecanique comme : 
	\begin{equation}\label{e_m_b_ii}
	E_{mec} = \frac{1}{2}m\left|\left|\Vec{v}\right|\right|^2 - qEz
	\end{equation}
	En d\'efinissant $\Delta E(t)$ comme $E(t)-E(0)$, on fait une \'etude de convergence de $\Delta E(t)$ en fonction de nombre de pas (nsteps). Ceci est repr\'esent\'e sur la fig \ref{Fig: Cons Emec}.
	
	\begin{figure}[h]
					\centering
					\includegraphics[scale = 0.6]{ener_mec_cons.png}
					\caption{\em\label{Fig: Cons Emec} $\Delta E(t)$ en fonction du nombre de pas de temps.}
	\end{figure}
\noindent On n'arrive pratiquement pas \`a diff\'erencier les droites pour les valeurs : $n_{steps} = 4642$, $n_{steps} = 6952$ et $n_{steps} = 10000$, mais on arrive bien \`a voir que la valeur de $\Delta E(t)$ semble bien converger lorsque $n_{steps}$ devient grand, on peut donc en d\'eduire que :

\begin{equation}\label{cons_en}
\lim_{n_{steps} \rightarrow +\infty} \Delta E(t) = 0 \forall t \in [0;t_{fin}]
\end{equation}

\subsubsection{Trajectoire de la particule dans un r\'ef\'erentiel en translation uniforme \`a la vitesse $v_E = \Vec{E}\times\Vec{B}/B^2$}

\noindent La formule de $v_E$ \'etant donn\'ee, on peut calculer sa valeur : 
\begin{equation}\label{v_e}
\Vec{v}_E = \frac{1}{B_0^2}\begin{pmatrix}0 \\ 0 \\ E_0 \end{pmatrix}\times\begin{pmatrix}0 \\ B_0 \\ 0 \end{pmatrix} = -\frac{1}{B_0}\begin{pmatrix}E_0 \\ 0 \\ 0 \end{pmatrix}
\end{equation}

\noindent On peut soustraire la valeur dans l'\'equation de $v_x(t)$ de l'\'equation \ref{v}, ce qui donne :
\begin{equation}\label{sol v_e}
v_x(t) = - v_0\sin\left(\frac{qB_0}{m}t\right) + \frac{E_0}{B_0}\cos\left(\frac{qB_0}{m}t\right)
\end{equation}

\begin{figure}[h]
				\begin{minipage}[c]{.46\linewidth}
					\centering
					\includegraphics[scale = 0.6]{traj_RK2_b_iii.png}
					\caption{\em\label{Fig: Traj R-K-2} La trajectoire de la particule dans le sch\'ema de Runge-Kutta d'ordre 2.}
				\end{minipage}
				\hfill%
				\begin{minipage}[c]{.46\linewidth}
					\centering
					\includegraphics[scale = 0.6]{traj_BB_b_iii.png}
					\caption{\em\label{Fig: Traj B-B} La trajectoire de la particule dans le sch\'ema de Boris-Buneman.}
				\end{minipage}
			\end{figure}

\noindent On voit sur la figure de Runge-Kutta (\ref{Fig: Traj R-K-2}) et sur la figure de Boris-Buneman (\ref{Fig: Traj B-B}) que les trajectoires forment une ellipse, ce qui est attendu, car la translation implique que le r\'ef\'erentiel "suit" la particule le long de l'axe $\hat{x}$.


\subsection{Cas d'un champ magn\'etique non uniforme : d\'erive $\Vec{B}\times\nabla B$}
\noindent Dans cette section nous consid\'erons un champ magn\'etique d'intensit\'e variable $\Vec{B} = B(x)\hat{y}$, o\`u $B(x) = B_0(1 + \kappa x)$, $\kappa = 100$m$^{-1}$, $B_0 = 5$T, $E = 0$, $v_{x_0} = 0$, $v_{z_0} = 4\cdot10^5$ms$^{-1}$, $x_0 = \frac{mv_{z_0}}{qB_0}$, $z_0 = 0$ et $t_{fin} = 6.5593\cdot10^{-7}$s \cite{Notes}. Le sch\'ema utilis\'e est celui de Boris-Buneman.

\subsubsection{Cas d'un proton ($q = e$) et d'un antiproton ($q = -e$)}
\noindent Dans les cas d'un proton et d'un antiproton, les trajectoires sont des cyclo\"\i des. On peut observer que lorsque $n_{steps}$ devient tr\`es grand, leurs orbites deviennent de plus en plus des colonnes droites. On peut voir leurs limites sur les figures \ref{Fig: Prot Traj} et \ref{Fig: Anti Traj}.
\begin{figure}[h]
				\begin{minipage}[c]{.46\linewidth}
					\centering
					\includegraphics[scale = 0.6]{final_traj_prot.png}
					\caption{\em\label{Fig: Prot Traj} Trajectoire d'un proton.}
				\end{minipage}
				\hfill%
				\begin{minipage}[c]{.46\linewidth}
					\centering
					\includegraphics[scale = 0.6]{final_traj_antiprot.png}
					\caption{\em\label{Fig: Anti Traj} Trajectoire d'un antiproton.}
				\end{minipage}
			\end{figure}
On voit que la colonne du proton est orient\'ee vers le bas, alors que celle de l'antiproton est orient\'ee vers le haut. La trajectoire est parcourue dans le sens anti-horaire par le proton, et dans le sens horaire par l'antiproton.

\subsubsection{La quantit\'e $\mu = \frac{mv^2}{2B}$}
\noindent La quantit\'e $\mu$ s'appelle le moment magn\'etique d'une particule, dans notre cas soit du proton, soit de l'antiproton. Il caract\'erise l'intensit\'e de la source magn\'etique. L'\'enonc\'e \cite{Notes} nous dit que $B(x) = B_0(1 + \kappa x) = B_0 + B_0x\kappa$. Il est \'evident que cette fonction d\'epend enti\`erement de la valeur de $x$, et a donc des extrema en fonction des extrema de la fonction $f(x) = x$. Sur les figures \ref{Fig: Pos Prot}, \ref{Fig: Pos Anti}, \ref{Fig: mm Prot} et \ref{Fig: mm Anti}, o\`u les positions des particules et les valeurs des moments magn\'etiques des particules sont repr\'esent\'ees sur un intervalle de temps $\Delta t = 1\times 10^{-7}$ s, pour qu'on puisse voir avec plus de pr\'ecision ce qui se passe, on voit que $\mu$ est inversement proportionnel \`a la position de la particule.

\begin{figure}
\begin{minipage}[c]{.46\linewidth}
					\centering
					\includegraphics[scale = 0.6]{proton_pos.png}
					\caption{\em\label{Fig: Pos Prot} Position du proton.}
				\end{minipage}
				\hfill%
				\begin{minipage}[c]{.46\linewidth}
					\centering
					\includegraphics[scale = 0.6]{antiproton_pos.png}
					\caption{\em\label{Fig: Pos Anti} Position du antiproton.}
				\end{minipage}
			\end{figure}
		
\begin{figure}
\begin{minipage}[c]{.46\linewidth}
					\centering
					\includegraphics[scale = 0.6]{mm_p.png}
					\caption{\em\label{Fig: mm Prot} Moment magn\'etique du proton.}
				\end{minipage}
				\hfill%
				\begin{minipage}[c]{.46\linewidth}
					\centering
					\includegraphics[scale = 0.6]{mm_ap.png}
					\caption{\em\label{Fig: mm Anti} Moment magn\'etique du antiproton.}
				\end{minipage}
			\end{figure}





\section{Conclusions}

 Apr\`es avoir fait cet exercice, on conna\^it mieux les int\'egrateurs et leurs sp\'ecificit\'es propres. 

 On remarque que la convergence de l'erreur sur la position et sur la vitesse sont du m\^eme ordre pour tous les int\'egrateurs \'etudi\'es. 
On peut voir que plus l'int\'egrateur est compliqu\'e, plus son erreur est basse et que tous ne convergent pas de la m\^eme mani\`ere lorsque $\Delta t$ devient plus grand. 

Quant \`a la conservation de l'\'energie m\'ecanique, il est visible que les sch\'emas d'Euler explicite et de Runge-Kutta d'ordre 2 sont les moins pr\'ecis.  Ensuite, Euler-Cromer converge vers la bonne valeur d\`es que le pas de temps minimum est respect\'e. De plus, Euler implicite est stable, mais il dissipe un peu d'\'energie \`a cause des erreurs num\'eriques. 
On remarque aussi que l'int\'egrateur le plus efficace est celui de Boris-Buneman car il conserve exactement l'\'energie m\'ecanique, aux arrondis pr\`es

%Ta partie
L'application \`a un cas o\`u le champ magn\'etique est variable, introduit le moment magn\'etique, qui caract\'erise l'intensit\'e du champ magn\'etique, et met en \'evidence le fait que celui-ci est inversement proportionnel \`a la position de la particule.


Ainsi, on a appris \`a conna\^itre plusieurs int\'egrateurs et on voit que pour des composantes diff\'erentes, les int\'egrateurs n'agissent pas froc\'ement de la m\^eme mani\`ere. On va pouvoir \^etre plus \`a m\^eme de choisir celui qui convient le mieux lors de futurs exercices. Il faut cependant savoir trouver l'\'equilibre entre l'erreur que l'on est pr\^et \`a avoir et la place et que l'on a pour calculer.


%-----------------------------------------------------------


\begin{thebibliography}{99}
 \bibitem{NdC}
 L. Villard avec la contribution de A. L\"auchli {\it Notes de cours Physique numérique I-II, version 20.1} (2020)
 
 \bibitem{Notes}
 L. Villard, Dr C. Sommariva {\it \'Enonc\'e de l'exercice 2} (2020)
 \url{https://moodle.epfl.ch/pluginfile.php/2839539/mod_resource/content/1/Exercice2_2020.pdf}
 
 \bibitem{FT}
 VEREIN SCHWEIZERISCHER MATHEMATIK- UND PHYSIKLEHRER  et al. {\it Formulaires et tables : mathématiques, physique, chimie} Editions G d'Encre, 2015. ISBN 978-2-940501-41-0
 
\end{thebibliography}

\end{document} %%%% THE END %%%%
